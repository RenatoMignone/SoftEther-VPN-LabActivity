\newpage

\section{Initial Network Configuration}

This section details the step-by-step process of setting up the laboratory environment, including GNS3 configuration, router setup, and Docker container deployment. The configuration establishes the basic network infrastructure that will support the VPN implementations.

\subsection{GNS3 Environment Setup}

The laboratory utilizes GNS3 (Graphical Network Simulator-3) to create a realistic network topology. GNS3 provides the platform for simulating Cisco routers, Docker containers, and network connectivity required for the VPN demonstration.

\subsubsection{Device Import and Template Creation}

\textbf{Cisco Router Template:}

\noindent
The topology requires Cisco 7200 routers running IOS version 124-24.T5. This specific IOS image is chosen because it represents a freely available alternative to modern commercial Cisco IOS versions. While newer platforms like IOSv (IOS Virtual) offer enhanced features and performance, they require valid Cisco licensing agreements which can be costly for educational purposes. The 124-24.T5 image, although older and not widely deployed in current production environments, provides all the fundamental routing and NAT capabilities required for this laboratory exercise. To import the router template:

\begin{enumerate}
    \item Open GNS3 and click the \textbf{router icon} in the left panel
    \item Select \textbf{"New Template"} at the bottom
    \item Choose \textbf{"Install an appliance from the GNS3 server"}
    \item Filter by \textbf{"CISCO"} and locate \textbf{"Cisco 7200"}
    \item Install the template and select version \textbf{"124-24.T5"}
    \item Import the router image file to complete the setup
\end{enumerate}

\noindent
\textbf{Docker Container Setup:}

\noindent
Two Docker containers are required for the VPN endpoints:

\begin{itemize}
    \item \textbf{Server Container Image:} \texttt{siomiz/softethervpn:latest}
    \item \textbf{Client Container Image:} \texttt{ubuntu:latest}
\end{itemize}

\noindent
To add containers to the GNS3 project:

\begin{enumerate}
    \item Navigate to Edit → Preferences in GNS3
    \item Select "Docker containers" from the left panel
    \item Click "New" to add a container
    \item Enter the image name (e.g., \texttt{siomiz/softethervpn:latest})
    \item Configure the container name and complete the setup
    \item Repeat for the Ubuntu client container
\end{enumerate}

\noindent
You will now see the containers in the left panel of GNS3, ready to be dragged into the workspace. So now you can recreate the same topology as shown in the figure in the section 3 of this document. Clicking on the wire on the left panel, you will be able to connect the devices together, creating the network topology defined. Choose the right interfaces for each connection as shown in the table in the section 3.2.\\

\noindent
\textbf{Important Router Configuration Note:}
\noindent
Before proceeding with the router configurations, you must configure the ISP router to have three interfaces instead of the default two. This is necessary because the ISP router needs to connect to both edge routers and the Internet cloud. To configure the ISP router slots:

\begin{enumerate}
    \item Right-click on the ISP router in GNS3
    \item Select \textbf{"Configure"} from the context menu
    \item Navigate to the \textbf{"Slots"} section in the configuration dialog
    \item Replace the existing slot configuration from \textbf{"C7200-IO-FE"} to \textbf{"C7200-IO-2FE"}
    \item Apply the changes and close the configuration dialog
\end{enumerate}

This configuration change provides the ISP router with three FastEthernet interfaces (Fa0/0, Fa0/1, and Fa1/0) as required by the network topology, allowing it to connect to both edge routers and the Internet cloud simultaneously.

\subsection{Container Configuration}

The Docker containers require specific configuration for persistent storage and network setup.

\subsubsection{Server Container Setup}

The SoftEther VPN server container needs persistent directories for configuration files:

\begin{enumerate}
    \item Open the server container configuration in GNS3
    \item Navigate to Advanced Settings
    \item Add the following additional directory:
    \begin{itemize}
        \item \texttt{/usr/vpnserver}
    \end{itemize}
\end{enumerate}

\noindent
This directory will be created in the GNS3 project folder, allowing persistent storage of configuration files between container restarts.\\

\noindent
\textbf{Server Configuration File Setup:}

\noindent
To ensure the SoftEther VPN server starts with the proper configuration, you must place the \texttt{vpn\_server.config} file in the correct persistent directory:

\begin{enumerate}
    \item Navigate to your GNS3 project directory on the local filesystem
    \item Open the \texttt{project-files} folder
    \item Locate the \texttt{docker} subdirectory
    \item Find the server container folder (usually named after the container)
    \item Navigate to the \texttt{usr/vpnserver} directory within the container folder
    \item Place the \texttt{vpn\_server.config} file in this directory
\end{enumerate}

\noindent
This ensures that when the container starts, the SoftEther VPN service will automatically load the pre-configured settings, including user accounts, IPSec settings, and SSL/TLS configurations.

\subsubsection{Client Container Setup}

The client container requires similar configuration for persistent storage:

\begin{enumerate}
    \item Open the client container configuration in GNS3
    \item Navigate to Advanced Settings  
    \item Add the additional directory: \texttt{/client}
\end{enumerate}

\noindent
\textbf{Client Configuration Files Setup:}

\noindent
The client container requires multiple configuration files for both IPSec and TLS VPN implementations. Place all the following files in the client's persistent directory:

\begin{enumerate}
    \item Navigate to your GNS3 project directory on the local filesystem
    \item Open the \texttt{project-files} folder
    \item Locate the \texttt{docker} subdirectory
    \item Find the client container folder
    \item Navigate to the \texttt{client} directory within the container folder
    \item Place the following files in this directory:
    \begin{itemize}
        \item \textbf{IPSec files:} \texttt{ipsec.conf} and \texttt{ipsec.secrets}
        \item \textbf{TLS/OpenVPN files:} \texttt{softether.ovpn}, \texttt{ca.crt}, and \texttt{credentials.txt}
    \end{itemize}
\end{enumerate}

\noindent
This setup ensures that all necessary VPN client configuration files are available when the container starts, allowing seamless connection to both IPSec and TLS VPN services.

\subsection{Project Startup and Device Access}

Once all configurations are complete and files are properly placed in their persistent directories, you can start the laboratory environment.

\subsubsection{Starting the GNS3 Project}

To start all devices and containers in the project:

\begin{enumerate}
    \item Click the green \textbf{Play button} in the top toolbar of GNS3
    \item This will start all routers and Docker containers simultaneously
    \item Wait for all devices to complete their boot process
    \item Verify that all device icons show a green status indicator
\end{enumerate}

\subsubsection{Accessing Device Terminals}

To interact with the network devices and containers:

\textbf{For Cisco Routers:}
\begin{enumerate}
    \item Right-click on any router device
    \item Select \textbf{"Console"} from the context menu
    \item This opens the router's command-line interface
\end{enumerate}

\noindent
\textbf{For Docker Containers:}
\begin{enumerate}
    \item Right-click on any container (server or client)
    \item Select \textbf{"Auxiliary Console"} from the context menu
    \item This opens the container's terminal interface
    \item Use this terminal for all container-based commands and configurations
\end{enumerate}

\textbf{Important Note:} For Docker containers, always use the "Auxiliary Console" option rather than the regular console, as it provides the proper terminal interface for interacting with the containerized operating system.

\subsection{Router Configuration}

Now that you can access the router terminals, proceed with configuring the three routers with specific configurations to simulate realistic Internet connectivity and routing behavior.

\subsubsection{ISP Router (R1-Router-ISP)}

The ISP router provides Internet connectivity and inter-site routing. The configuration includes interface setup, MAC address spoofing for DHCP, and routing table entries.

\begin{lstlisting}[language=bash]
enable
configure terminal

# Configure interface to Router 2 (Server-side)
interface FastEthernet0/0
  ip address 203.0.113.254 255.255.255.0
  no shutdown
exit

# Configure interface to Router 3 (Client-side)  
interface FastEthernet0/1
  ip address 198.51.100.254 255.255.255.0
  no shutdown
exit

# Configure Internet interface with DHCP and MAC spoofing
interface FastEthernet1/0
  mac-address xxxx.xxxx.xxxx  # Replace with host machine MAC
  ip address dhcp
  no shutdown
exit

# Configure static routes for private networks
ip route 198.51.100.0 255.255.255.0 FastEthernet0/1
ip route 203.0.113.0 255.255.255.0 FastEthernet0/0

# Configure default route to Internet
ip route 0.0.0.0 0.0.0.0 FastEthernet1/0

end
write memory
\end{lstlisting}

\noindent
\textbf{Important Note - MAC Address Spoofing:} 

\noindent
The MAC address spoofing is necessary because the Politecnico di Torino WiFi network's DHCP server implements MAC address filtering for security purposes. Without using the host machine's MAC address, the DHCP server will not assign a valid IP address to the router interface, preventing Internet connectivity for the simulated network. To obtain your host machine's MAC address:

\begin{lstlisting}[language=bash]
# Display all network interfaces and their MAC addresses
ip a

# Look for the active network interface (usually wlan0 for WiFi or eth0 for Ethernet)
# The MAC address appears after "link/ether"
# Example output:
# 2: wlan0: <BROADCAST,MULTICAST,UP,LOWER_UP> mtu 1500 qdisc mq state UP group default qlen 1000
#     link/ether 1c:ce:51:3f:98:20 brd ff:ff:ff:ff:ff:ff
\end{lstlisting}

\noindent
Copy the MAC address from your active network interface and convert it to Cisco format by adding dots every four characters:
\begin{itemize}
    \item \textbf{Linux format:} 1c:ce:51:3f:98:20
    \item \textbf{Cisco format:} 1cce.513f.9820
\end{itemize}

Replace \texttt{xxxx.xxxx.xxxx} in the configuration with your converted MAC address.

\subsubsection{Server-Side Router (R2-Router-Server)}

This router connects the server's private network to the Internet and implements NAT with port forwarding for VPN services.

\begin{lstlisting}[language=bash]
enable
configure terminal

# Configure LAN interface (connected to server)
interface FastEthernet0/0
  ip address 10.0.1.1 255.255.255.0
  ip nat inside
  no shutdown
exit

# Configure WAN interface (connected to ISP)
interface FastEthernet0/1
  ip address 203.0.113.1 255.255.255.0
  ip nat outside
  no shutdown
exit

# Configure static NAT for VPN services
ip nat inside source static udp 10.0.1.2 500 203.0.113.1 500
ip nat inside source static udp 10.0.1.2 4500 203.0.113.1 4500
ip nat inside source static tcp 10.0.1.2 443 203.0.113.1 443

# Configure PAT for general Internet access
access-list 1 permit 10.0.1.0 0.0.0.255
ip nat inside source list 1 interface FastEthernet0/1 overload

# Configure default route
ip route 0.0.0.0 0.0.0.0 FastEthernet0/1

end
write memory
\end{lstlisting}

\noindent
\textbf{Port Forwarding Explanation:}
\begin{itemize}
    \item \textbf{Port 500/UDP:} ISAKMP (Internet Security Association and Key Management Protocol)
    \item \textbf{Port 4500/UDP:} NAT-T (NAT Traversal for IPSec)
    \item \textbf{Port 443/TCP:} HTTPS/TLS for SSL VPN connectivity
\end{itemize}

\subsubsection{Client-Side Router (R3-Router-Client)}

The client-side router provides NAT and routing for the client's private network.

\begin{lstlisting}[language=bash]
enable
configure terminal

# Configure LAN interface (connected to client)
interface FastEthernet0/0
  ip address 10.0.2.1 255.255.255.0
  ip nat inside
  no shutdown
exit

# Configure WAN interface (connected to ISP)
interface FastEthernet0/1
  ip address 198.51.100.1 255.255.255.0
  ip nat outside
  no shutdown
exit

# Configure default route
ip route 0.0.0.0 0.0.0.0 FastEthernet0/1

# Configure PAT for Internet access
access-list 1 permit 10.0.2.0 0.0.0.255
ip nat inside source list 1 interface FastEthernet0/1 overload

end
write memory
\end{lstlisting}

\subsection{Container Network Configuration}

Now that you have access to the container terminals, configure the network interfaces for both containers.

\subsubsection{Server Network Configuration}

Configure the server container's network interface:

\begin{lstlisting}[language=bash]
# Configure IP address and default route
ip addr add 10.0.1.2/24 dev eth0
ip route add default via 10.0.1.1

# Verify network configuration
ip addr show
ip route show

# Test connectivity to gateway
ping 10.0.1.1
\end{lstlisting}

\subsubsection{Client Network Configuration}

Configure the client container's network interface and install required VPN software:

\begin{lstlisting}[language=bash]
# Configure network interface
ip addr add 10.0.2.2/24 dev eth0
ip route add default via 10.0.2.1

# Update package repositories
apt update

# Install strongSwan for IPSec VPN
apt install strongswan -y

# Install OpenVPN for TLS VPN  
apt install openvpn -y

# Copy configuration files from persistent storage
cp /client/ipsec.conf /etc/
cp /client/ipsec.secrets /etc/

# Verify network configuration
ip addr show
ip route show

# Test connectivity to gateway
ping 10.0.2.1
\end{lstlisting}

\subsection{Basic Connectivity Testing}

Before proceeding with VPN configuration, verify that the basic network infrastructure is functioning correctly.

\subsubsection{Inter-Router Connectivity}

Test connectivity between routers to ensure proper routing:

\begin{lstlisting}[language=bash]
# From ISP Router - test connectivity to edge routers
ping 203.0.113.1    # Should reach Server-side router
ping 198.51.100.1   # Should reach Client-side router

# From Server-side Router - test connectivity to ISP
ping 203.0.113.254  # Should reach ISP router

# From Client-side Router - test connectivity to ISP  
ping 198.51.100.254 # Should reach ISP router
\end{lstlisting}

\subsubsection{End-to-End Connectivity}

Test connectivity between the container endpoints:

\begin{lstlisting}[language=bash]
# From Server Container - test connectivity to public IPs
ping 198.51.100.1   # Should reach Client-side router public IP

# From Client Container - test connectivity to public IPs
ping 203.0.113.1    # Should reach Server-side router public IP
\end{lstlisting}

\subsubsection{Service Verification}

Verify that the SoftEther VPN server is running and listening on required ports:

\begin{lstlisting}[language=bash]
# Check if SoftEther VPN server is listening
ss -tuln | grep -E '(443|500|4500)'

# Expected output should show:
# tcp LISTEN 0.0.0.0:443
# udp LISTEN 0.0.0.0:500  
# udp LISTEN 0.0.0.0:4500
\end{lstlisting}

\subsection{Network Infrastructure Validation}

At this point, the basic network infrastructure should be operational with:

\begin{itemize}
    \item Three routers configured with appropriate IP addressing and routing
    \item NAT services functioning on edge routers
    \item Docker containers with network connectivity
    \item SoftEther VPN server operational and accessible
    \item Basic inter-site connectivity through the simulated Internet
\end{itemize}

\noindent
This foundation enables the implementation of VPN services detailed in the subsequent sections. Any connectivity issues at this stage should be resolved before proceeding with VPN configuration, as they will prevent proper VPN tunnel establishment.

\noindent
The next section will detail the configuration of the SoftEther VPN server to provide multi-protocol VPN services for both IPSec and TLS-based connections.
