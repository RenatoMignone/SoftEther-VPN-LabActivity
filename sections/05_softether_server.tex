\newpage

\section{SoftEther VPN Server Configuration}

This section explains the deployment and installation of the SoftEther VPN server within the Docker container environment. 

\subsection{Server Setup and Installation}

The SoftEther VPN server runs in a Docker container from the pre-built image \texttt{siomiz/softethervpn:latest}. This containerized approach has various advantages like isolation, portability, and simplicity of deployment.

\subsubsection{Container Deployment}

The server container has been configured in GNS3 with persistent storage folders as explained in Section 4. Once the container is running, the SoftEther VPN service automatically starts with the following features:

\begin{itemize}
    \item \textbf{Automatic Service Start:} The VPN server daemon automatically runs when the container starts, due to the persistent storage setup, the server will be already configured with the correct configuration file.
    \item \textbf{Multi-Protocol Listeners:} Support for IPSec, SSL/TLS simultaneously
    \item \textbf{Network Integration:} Automatic integration with the container's network interface (eth0)
\end{itemize}

\subsubsection{Service Verification}

To verify that the SoftEther VPN server operates correctly, execute the following commands from within the server container:

\begin{lstlisting}[language=bash]
# Check if SoftEther VPN server process is running
ps aux | grep vpnserver

# Verify VPN server is listening on required ports
ss -tuln | grep -E '(443|500|4500|5555)'

# Expected output should show listeners on:
# tcp LISTEN 0.0.0.0:443   (HTTPS/SSL VPN)
# udp LISTEN 0.0.0.0:500   (ISAKMP/IKE)
# udp LISTEN 0.0.0.0:4500  (NAT-T)
# tcp LISTEN 0.0.0.0:5555  (Management/API)
\end{lstlisting}

\subsection{Configuration File Analysis}

The SoftEther VPN server configuration is configured by the \texttt{vpn\_server.config} file, with detailed settings for all VPN protocols, virtual hubs, user administration, and security policies.

\subsubsection{Core Server Settings}

Server setup consists of several core elements

\begin{lstlisting}[language=bash]
# Core server configuration parameters
declare ServerConfiguration
{
    # Accept non-TLS connections
    bool AcceptOnlyTls false    
    # Default cipher suite               
    string CipherName DHE-RSA-AES256-SHA       
    # Allow IPSec aggressive mode
    bool DisableIPsecAggressiveMode false      
    # Enable NAT traversal
    bool DisableNatTraversal false             
    # Enable OpenVPN compatibility
    bool DisableOpenVPNServer false            
    # Connection limit per IP
    uint MaxConnectionsPerIP 256               
    # Enable debug logging
    bool SaveDebugLog true                     
}
\end{lstlisting}

\textbf{Key Configuration Parameters:}

\begin{itemize}
    \item \textbf{Multi-Protocol Support:} All significant VPN protocols are supported by default
    \item \textbf{NAT Traversal:} Should be enabled for clients behind NAT devices
    \item \textbf{Security Settings:} Secure cipher suites with DHE for perfect forward secrecy
    \item \textbf{Connection Limits:} Reasonable connection limits to prevent resource exhaustion
    \item \textbf{Logging:} Comprehensive logging for troubleshooting and analysis
\end{itemize}

\subsubsection{Protocol Listener Configuration}

The server establishes multiple listeners for a range of VPN protocols:

\begin{lstlisting}[language=bash]
declare ListenerList
{
    declare Listener0  # HTTPS/SSL VPN
    {
        bool Enabled true
        uint Port 443
    }
    declare Listener1  # ISAKMP/IKE
    {
        bool Enabled true
        uint Port 500
    }
    declare Listener2  # NAT-T
    {
        bool Enabled true
        uint Port 4500
    }
    declare Listener3  # Management/API
    {
        bool Enabled true
        uint Port 5555
    }
}
\end{lstlisting}

\textbf{Protocol Listener Details:}

\begin{itemize}
    \item \textbf{Port 443/TCP:} HTTPS/SSL VPN for OpenVPN-compatible connections
    \item \textbf{Port 500/UDP:} ISAKMP/IKE for initial IPSec negotiation
    \item \textbf{Port 4500/UDP:} NAT-T (NAT Traversal) - Essential for IPSec operation behind NAT device.
    \item \textbf{Port 5555/TCP:} Management and API interface for server administration
\end{itemize}

\subsection{Virtual Hub and User Management}

SoftEther VPN handles connections with Virtual Hubs, which act as virtual Ethernet switches. The default setting includes a single hub named "DEFAULT" with basic user authentication.

\subsubsection{Default Virtual Hub Configuration}

The DEFAULT hub provides the following services:

\begin{itemize}
    \item \textbf{User Authentication:} basic Password-based authentication for VPN clients (not using the cerificate authentication for the mutual authentication)
    \item \textbf{SecureNAT:}  In-built DHCP and NAT services for allocating IP to clients
    \item \textbf{Access Control:} Traffic filtering and routing policies can be configured
\end{itemize}

\subsubsection{User Account Management}

The server also includes a test user account for VPN authentication:

\begin{lstlisting}[language=bash]
declare UserList
{
    declare user1
    {
        # Password: "ciao" (hashed)
        byte AuthPassword ObNWU1DckHL0Xg4HuyRAMKiIANY= 
        # Password authentication   
        uint AuthType 1                    
        # No additional notes              
        string Note $                                                    
    }
}
\end{lstlisting}

\textbf{Important Authentication Note:} The user credentials (user1/ciao) are not used during the TLS handshake process itself. Instead, the credentials are used for an application-layer authentication that is performed after successful establishment of the TLS tunnel. The TLS handshake only authenticates the server certificate, while the user authentication happens at a higher layer within the securely established tunnel.

\subsection{IPSec Protocol Configuration}

The SoftEther VPN server already has IPSec built-in, so a normal IPSec client can be connected without extra software. Pre-shared key is not a strong one here, but it works for lab environments.

\subsubsection{IPSec Settings}

\begin{lstlisting}[language=bash]
declare IPsec
{               
    # Pre-shared key for IPSec  
    string IPsec_Secret ciao                   
    # Default hub for L2TP connections
    string L2TP_DefaultHub DEFAULT    
    # Disable L2TP/IPSec (using native IPSec)         
    bool L2TP_IPsec false                      
    # Disable raw L2TP
    bool L2TP_Raw false                        
}
\end{lstlisting}

\textbf{IPSec Configuration Details:}

\begin{itemize}
    \item \textbf{Pre-Shared Key:} "ciao" - used for IPSec authentication, usage of a weak key only for lab purposes
    \item \textbf{Default Hub:} All IPSec connections are directed to the DEFAULT virtual hub
    \item \textbf{Protocol Mode:} Native IPSec implementation rather than L2TP/IPSec combination
\end{itemize}

\subsection{SSL/TLS Protocol Configuration}

The server provides SSL/TLS VPN services compatible with OpenVPN clients and other SSL VPN solutions.

\subsubsection{TLS Settings and Certificates}

The server uses self-signed certificates for TLS connections:

\begin{lstlisting}[language=bash]
# TLS configuration parameters
string OpenVPNDefaultClientOption dev-type$20tun,link-mtu$201500,tun-mtu$201500,cipher$20AES-128-CBC,auth$20SHA1,keysize$20128,key-method$202,tls-client
\end{lstlisting}

\noindent
\textbf{SSL/TLS Features:}

\begin{itemize}
    \item \textbf{OpenVPN Compatibility:} Full compatibility with standard OpenVPN clients
    \item \textbf{Cipher Support:} AES-128-CBC encryption with SHA1 authentication. Usage of a weak hash function only for lab purposes, to increase the security, it is recommended to use SHA256 or stronger, modifying accordingly the other parameters.
    \item \textbf{TUN Interface:} Layer-3 tunneling for IP packet forwarding
    \item \textbf{Certificate-Based Authentication:} X.509 certificate validation for enhanced security
\end{itemize}

\subsection{SecureNAT Configuration}

SecureNAT provides integrated DHCP and NAT services for VPN clients, simplifying client configuration and enabling Internet access.

\textbf{SecureNAT Benefits:}

\begin{itemize}
    \item \textbf{Automatic IP Assignment:} Clients receive IP addresses from 192.168.30.10-200 range
    \item \textbf{DNS Services:} Integrated DNS resolution for VPN clients
    \item \textbf{Internet Access:} NAT functionality enables clients to access external resources
    \item \textbf{Simplified Configuration:} Clients require minimal manual network configuration
\end{itemize}

\subsubsection{Server Certificates}

The server uses self-signed X.509 certificates for TLS operations:

\begin{itemize}
    \item \textbf{Certificate Subject:} da3af5075c51 (unique identifier)
    \item \textbf{Key Length:} 2048-bit RSA
    \item \textbf{Validity Period:} Long-term validity for lab use
    \item \textbf{Usage:} Server authentication for TLS/SSL connections
\end{itemize}

\subsubsection{Security Policies}

The server uses numerous security policies:

\begin{lstlisting}[language=bash]
# Security-related settings
# Enable DoS protection
bool DisableDosProtection false                     
# Allow session reconnection
bool DisableSessionReconnect false               
# DNS thread limit 
uint MaxConcurrentDnsClientThreads 512           
# Connection limit
uint MaxUnestablishedConnections 1000            
# Send server signature
bool NoSendSignature false                       
\end{lstlisting}

\noindent
\textbf{Security Features:}

\begin{itemize}
    \item \textbf{DoS Protection:} Built-in protection against denial-of-service attacks
    \item \textbf{Connection Limits:} Prevents resource exhaustion through connection limiting
    \item \textbf{Session Management:} Secure session handling with reconnection support
    \item \textbf{Authentication:} Multiple authentication methods including certificates and passwords
\end{itemize}

\subsection{Operational Verification}

Ensure the server configuration supports both IPSec and SSL/TLS protocols:

\begin{itemize}
    \item \textbf{IPSec Listeners:} Ports 500 (ISAKMP) and 4500 (NAT-T) are active
    \item \textbf{SSL/TLS Listeners:} Ports 443 is accepting connections 
    \item \textbf{User Authentication:} Test user "user1" is configured and accessible
    \item \textbf{Virtual Hub:} DEFAULT hub is operational with SecureNAT enabled
    \item \textbf{NAT Forwarding:} Router forwarding rules are directing traffic correctly
\end{itemize}

The SoftEther VPN server is configured and ready to listen for IPSec and TLS/SSL-connections. The sections that follow describe the configuration of both of these client types and show secure tunnel establishment over the test Internet infrastructure.
