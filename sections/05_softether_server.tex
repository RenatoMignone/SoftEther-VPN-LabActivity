\newpage

\section{SoftEther VPN Server Configuration}

This section details the configuration and deployment of the SoftEther VPN server within the Docker container environment. The server provides multi-protocol VPN services, supporting both IPSec and TLS/SSL connections simultaneously through a unified management interface.

\subsection{Server Installation and Initialization}

The SoftEther VPN server runs within a Docker container using the pre-built image \texttt{siomiz/softethervpn:latest}. This containerized approach provides several advantages including isolation, portability, and simplified deployment.

\subsubsection{Container Deployment}

The server container has been configured in GNS3 with persistent storage directories as detailed in Section 4. When the container starts, the SoftEther VPN service initializes automatically with the following characteristics:

\begin{itemize}
    \item \textbf{Automatic Service Start:} The VPN server daemon starts automatically when the container boots, thanks to the persistent storage configuration, the server will start already with the proper configuration file.
    \item \textbf{Multi-Protocol Listeners:} Simultaneous support for IPSec, SSL/TLS
    \item \textbf{Network Integration:} Automatic integration with the container's network interface (eth0)
\end{itemize}

\subsubsection{Service Verification}

To verify the SoftEther VPN server is running correctly, execute the following commands within the server container:

\begin{lstlisting}[language=bash]
# Check if SoftEther VPN server process is running
ps aux | grep vpnserver

# Verify VPN server is listening on required ports
ss -tuln | grep -E '(443|500|4500|5555)'

# Expected output should show listeners on:
# tcp LISTEN 0.0.0.0:443   (HTTPS/SSL VPN)
# udp LISTEN 0.0.0.0:500   (ISAKMP/IKE)
# udp LISTEN 0.0.0.0:4500  (NAT-T)
# tcp LISTEN 0.0.0.0:5555  (Management/API)
\end{lstlisting}

\subsection{Configuration File Analysis}

The SoftEther VPN server configuration is managed through the \texttt{vpn\_server.config} file, which contains comprehensive settings for all VPN protocols, virtual hubs, user management, and security policies.

\subsubsection{Core Server Settings}

The server configuration includes several critical components:

\begin{lstlisting}[language=bash]
# Core server configuration parameters
declare ServerConfiguration
{
    # Accept non-TLS connections
    bool AcceptOnlyTls false    
    # Default cipher suite               
    string CipherName DHE-RSA-AES256-SHA       
    # Allow IPSec aggressive mode
    bool DisableIPsecAggressiveMode false      
    # Enable NAT traversal
    bool DisableNatTraversal false             
    # Enable OpenVPN compatibility
    bool DisableOpenVPNServer false            
    # Connection limit per IP
    uint MaxConnectionsPerIP 256               
    # Enable debug logging
    bool SaveDebugLog true                     
}
\end{lstlisting}

\textbf{Key Configuration Parameters:}

\begin{itemize}
    \item \textbf{Multi-Protocol Support:} All major VPN protocols are enabled by default
    \item \textbf{NAT Traversal:} Essential for clients behind NAT devices
    \item \textbf{Security Settings:} Strong cipher suites with DHE for perfect forward secrecy
    \item \textbf{Connection Limits:} Reasonable limits to prevent resource exhaustion
    \item \textbf{Logging:} Comprehensive logging for troubleshooting and analysis
\end{itemize}

\subsubsection{Protocol Listener Configuration}

The server configures multiple listeners for different VPN protocols:

\begin{lstlisting}[language=bash]
declare ListenerList
{
    declare Listener0  # HTTPS/SSL VPN
    {
        bool Enabled true
        uint Port 443
    }
    declare Listener1  # ISAKMP/IKE
    {
        bool Enabled true
        uint Port 500
    }
    declare Listener2  # NAT-T
    {
        bool Enabled true
        uint Port 4500
    }
    declare Listener3  # Management/API
    {
        bool Enabled true
        uint Port 5555
    }
}
\end{lstlisting}

\subsection{Virtual Hub and User Management}

SoftEther VPN organizes connections through Virtual Hubs, which act as virtual Ethernet switches. The default configuration includes a single hub named "DEFAULT" with basic user authentication.

\subsubsection{Default Virtual Hub Configuration}

The DEFAULT hub provides the following services:

\begin{itemize}
    \item \textbf{User Authentication:} basic Password-based authentication for VPN clients (not using the certificate authentication for the mutual authentication because of the lab environment)
    \item \textbf{SecureNAT:} Integrated DHCP and NAT services for client IP assignment
    \item \textbf{Access Control:} Configurable policies for traffic filtering and routing
    \item \textbf{Logging:} Comprehensive logging of user sessions and traffic
\end{itemize}

\subsubsection{User Account Management}

The server includes a test user account for VPN authentication:

\begin{lstlisting}[language=bash]
declare UserList
{
    declare user1
    {
        # Password: "ciao" (hashed)
        byte AuthPassword ObNWU1DckHL0Xg4HuyRAMKiIANY= 
        # Password authentication   
        uint AuthType 1                    
        # No additional notes              
        string Note $                                                    
    }
}
\end{lstlisting}

\subsection{IPSec Protocol Configuration}

The SoftEther VPN server includes native IPSec support, allowing standard IPSec clients to connect without additional software. The pre-shared key is not a strong one here, but it is suitable for laboratory use.

\subsubsection{IPSec Settings}

\begin{lstlisting}[language=bash]
declare IPsec
{               
    # Pre-shared key for IPSec  
    string IPsec_Secret ciao                   
    # Default hub for L2TP connections
    string L2TP_DefaultHub DEFAULT    
    # Disable L2TP/IPSec (using native IPSec)         
    bool L2TP_IPsec false                      
    # Disable raw L2TP
    bool L2TP_Raw false                        
}
\end{lstlisting}

\textbf{IPSec Configuration Details:}

\begin{itemize}
    \item \textbf{Pre-Shared Key:} "ciao" - used for IPSec authentication
    \item \textbf{Default Hub:} All IPSec connections are directed to the DEFAULT virtual hub
    \item \textbf{Protocol Mode:} Native IPSec implementation rather than L2TP/IPSec combination
\end{itemize}

\subsection{SSL/TLS Protocol Configuration}

The server provides SSL/TLS VPN services compatible with OpenVPN clients and other SSL VPN solutions.

\subsubsection{TLS Settings and Certificates}

The server uses self-signed certificates for TLS connections:

\begin{lstlisting}[language=bash]
# TLS configuration parameters
string OpenVPNDefaultClientOption dev-type$20tun,link-mtu$201500,tun-mtu$201500,cipher$20AES-128-CBC,auth$20SHA1,keysize$20128,key-method$202,tls-client
\end{lstlisting}

\textbf{SSL/TLS Features:}

\begin{itemize}
    \item \textbf{OpenVPN Compatibility:} Full compatibility with standard OpenVPN clients
    \item \textbf{Cipher Support:} AES-128-CBC encryption with SHA1 authentication
    \item \textbf{TUN Interface:} Layer-3 tunneling for IP packet forwarding
    \item \textbf{Certificate-Based Authentication:} X.509 certificate validation for enhanced security
\end{itemize}

\subsection{SecureNAT Configuration}

SecureNAT provides integrated DHCP and NAT services for VPN clients, simplifying client configuration and enabling Internet access.

\textbf{SecureNAT Benefits:}

\begin{itemize}
    \item \textbf{Automatic IP Assignment:} Clients receive IP addresses from 192.168.30.10-200 range
    \item \textbf{DNS Services:} Integrated DNS resolution for VPN clients
    \item \textbf{Internet Access:} NAT functionality enables clients to access external resources
    \item \textbf{Simplified Configuration:} Clients require minimal manual network configuration
\end{itemize}

\subsubsection{Server Certificates}

The server uses self-signed X.509 certificates for TLS operations:

\begin{itemize}
    \item \textbf{Certificate Subject:} da3af5075c51 (unique identifier)
    \item \textbf{Key Length:} 2048-bit RSA
    \item \textbf{Validity Period:} Long-term validity for laboratory use
    \item \textbf{Usage:} Server authentication for TLS/SSL connections
\end{itemize}

\subsubsection{Security Policies}

The server implements several security policies:

\begin{lstlisting}[language=bash]
# Security-related settings
# Enable DoS protection
bool DisableDosProction false                     
# Allow session reconnection
bool DisableSessionReconnect false               
# DNS thread limit 
uint MaxConcurrentDnsClientThreads 512           
# Connection limit
uint MaxUnestablishedConnections 1000            
# Send server signature
bool NoSendSignature false                       
\end{lstlisting}

\textbf{Security Features:}

\begin{itemize}
    \item \textbf{DoS Protection:} Built-in protection against denial-of-service attacks
    \item \textbf{Connection Limits:} Prevents resource exhaustion through connection limiting
    \item \textbf{Session Management:} Secure session handling with reconnection support
    \item \textbf{Authentication:} Multiple authentication methods including certificates and passwords
\end{itemize}

\subsection{Operational Verification}

After configuration, verify the server is operating correctly and ready to accept VPN connections.

\subsubsection{Service Status Check}

\begin{lstlisting}[language=bash]
# Verify all VPN protocols are listening
netstat -tuln | grep -E '(443|500|4500|992|5555)'

# Test connectivity from client network
# From client container:
ping 203.0.113.1    # Should reach server's public IP
telnet 203.0.113.1 443  # Should connect to HTTPS port
\end{lstlisting}

\subsubsection{Configuration Validation}

Ensure the server configuration supports both IPSec and SSL/TLS protocols:

\begin{itemize}
    \item \textbf{IPSec Listeners:} Ports 500 (ISAKMP) and 4500 (NAT-T) are active
    \item \textbf{SSL/TLS Listeners:} Ports 443 and 992 are accepting connections
    \item \textbf{User Authentication:} Test user "user1" is configured and accessible
    \item \textbf{Virtual Hub:} DEFAULT hub is operational with SecureNAT enabled
    \item \textbf{NAT Forwarding:} Router forwarding rules are directing traffic correctly
\end{itemize}

The SoftEther VPN server is now configured and ready to accept connections from both IPSec and SSL/TLS VPN clients. The next sections will detail the configuration of these client types and demonstrate secure tunnel establishment across the simulated Internet infrastructure.
