\newpage

\section{Conclusion}

This lab activity successfully demonstrated the implementation and comparative testing of Virtual Private Network technologies using SoftEther VPN's multi-protocol capability. Under hands-on configuration and testing, we established secure communication paths between geographically separated networks over a simulation Internet infrastructure.

\subsection{Major Achievements}

The lab was successful in its primary objectives:

\begin{itemize}
    \item \textbf{Network Infrastructure:} Successfully deployed an actual GNS3 topology of three Cisco routers, Docker containers, and correct NAT configuration among them to simulate Internet-based VPN environments
    
    \item \textbf{Multi-Protocol Implementation:} Configure both IPSec (with strongSwan) and TLS/SSL (with OpenVPN) VPN connections to one single SoftEther VPN server, demonstrating the plat-form's flexibility
    
    \item \textbf{Traffic Analysis:} Used Wireshark to capture VPN setup stages and data transfer over VPN in encrypted mode, tracking ISAKMP/ESP packets for IPSec and SSL/TLS records for OpenVPN
\end{itemize}

\subsection{Learning Outcomes}

The lab provided experiential learning on:

\begin{itemize}
    \item \textbf{Network Simulation:} Experiential learning in the usage of GNS3 for complex network topology configuration and management
    
    \item \textbf{VPN Technologies:}  Understanding how different VPN protocols establish secure tunnels and encapsulate traffic

    \item \textbf{Security Analysis:} Understanding packet capture and analysis tools for VPN security property testing
    
    \item \textbf{Container Deployment:} Understanding Docker container deployment for network service deployment

\end{itemize}


\subsection{Practical Insights}

The lab placed focus on prime concerns of real-world VPN deployments:

\begin{itemize}
    \item \textbf{Protocol Selection:} Whether to implement IPSec or TLS depends on factors such as existing infrastructure, client capabilities, performance requirements, and ease of installation
    
    \item \textbf{NAT Compatibility:} TLS-based solutions have better compatibility with NAT devices and restrictive firewall environments
    
    \item \textbf{Security Properties:} Both protocols implement the fundamental VPN security properties (confidentiality, integrity, authentication, anti-replay protection) but in a different manner and at a different network layer
    
\end{itemize}
