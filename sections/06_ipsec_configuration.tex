\newpage

\section{IPSec VPN Configuration}

This section details the configuration of the IPSec-based VPN connection using strongSwan as the client software. The IPSec implementation provides network-layer security with encryption and authentication, establishing a secure tunnel between the client container and the SoftEther VPN server.

\subsection{strongSwan Client Setup}

strongSwan is a complete IPSec implementation for Linux systems that supports both IKEv1 and IKEv2 protocols. It integrates with the Linux kernel's IPSec stack to provide high-performance encrypted tunnels.

\subsubsection{Software Installation}

The strongSwan software was installed during the initial container configuration as part of Section 4. To verify the installation and ensure all components are available:

\begin{lstlisting}[language=bash]
# Verify strongSwan installation
apt list --installed | grep strongswan

# Check available strongSwan tools
which ipsec
ipsec version

\end{lstlisting}

\subsection{IPSec Configuration Files}

strongSwan uses two primary configuration files: \texttt{ipsec.conf} for connection parameters and \texttt{ipsec.secrets} for authentication credentials.

\subsubsection{ipsec.conf Analysis}

The \texttt{ipsec.conf} file defines the IPSec connection parameters, including encryption algorithms, authentication methods, and tunnel endpoints.

\begin{lstlisting}[language=bash]
# /etc/ipsec.conf
config setup
    charondebug="ike 2, knl 2, cfg 2"   # Debugging levels
    uniqueids=yes                       # Ensure unique IDs

conn softether
    # Local settings
    left=%any                          # Accept any local IP
    leftsubnet=10.0.2.0/24             # Client subnet to encrypt
    leftid=@client                     # Unique client identifier
    
    # Remote settings  
    right=203.0.113.1                  # Server public IP
    rightid=10.0.1.2                   # Server's actual IP
    rightsubnet=0.0.0.0/0              # All traffic via VPN
    
    # NAT-Traversal
    forceencaps=yes                     # Force UDP encapsulation
    
    # Phase 1 (IKEv1)
    keyexchange=ikev1                   # Use IKEv1 protocol
    ike=aes256-sha1-modp2048!           # IKE encryption
    esp=aes128-sha1-modp1024!           # ESP encryption
    aggressive=no                       # Use main mode
    
    # Authentication
    leftauth=psk                        # Pwd authentication
    rightauth=psk                       # Server uses PSK too
    
    # Dead Peer Detection
    dpdaction=restart                   # Restart on peer failure
    dpddelay=30s                        # DPD check interval
    dpdtimeout=120s                     # DPD timeout
    
    auto=start                          # Auto-start connection
\end{lstlisting}

\textbf{Configuration Parameter Explanation:}

\begin{itemize}
    \item \textbf{Connection Identity:} \texttt{left/right} parameters define local and remote endpoints
    \item \textbf{Subnet Configuration:} \texttt{leftsubnet} specifies which traffic should be encrypted
    \item \textbf{NAT Traversal:} \texttt{forceencaps=yes} ensures IPSec works through NAT devices
    \item \textbf{Algorithm Selection:} Encryption with AES-256 for IKE and AES-128 for ESP
    \item \textbf{Authentication:} Pre-shared key (PSK) method matching server configuration
    \item \textbf{Dead Peer Detection:} Automatic tunnel recovery on connection failure
\end{itemize}

\subsubsection{ipsec.secrets Configuration}

The \texttt{ipsec.secrets} file contains authentication credentials, specifically the pre-shared key that must match the server configuration.

\begin{lstlisting}[language=bash]
# /etc/ipsec.secrets
# Format: local_id remote_id : PSK "shared_secret"

%any 203.0.113.1 : PSK "ciao"
\end{lstlisting}

\textbf{Authentication Details:}

\begin{itemize}
    \item \textbf{Local ID:} \texttt{\%any} accepts any local IP address
    \item \textbf{Remote ID:} \texttt{203.0.113.1} specifies the SoftEther server's public IP
    \item \textbf{Authentication Method:} PSK (Pre-Shared Key)
    \item \textbf{Shared Secret:} "ciao" - must match the server's IPSec\_Secret configuration
\end{itemize}

\noindent
\textbf{Important Security Note:} In production environments, pre-shared keys should be significantly more complex and randomly generated. The simple "ciao" key is used here for laboratory demonstration purposes only.

\subsection{Connection Establishment}

The IPSec tunnel establishment process involves multiple phases, including IKE negotiation and ESP Security Association creation.

\subsubsection{Manual Connection Initiation}

To establish the IPSec connection manually:

\begin{lstlisting}[language=bash]
# Restart strongSwan to reload configuration
ipsec restart

# Initiate the VPN connection
ipsec up softether

# Verify connection status
ipsec status

# Check established Security Associations
ipsec statusall
\end{lstlisting}

\subsubsection{Connection Verification}

Successful IPSec tunnel establishment can be verified through multiple methods:

\begin{lstlisting}[language=bash]
# Test connectivity through tunnel
ping 192.168.30.1  # SoftEther SecureNAT gateway

# Check routing table for VPN routes
ip route show

\end{lstlisting}

\subsubsection{Debug and Troubleshooting}

For troubleshooting connection issues, strongSwan provides comprehensive debugging:

\begin{lstlisting}[language=bash]
# Start strongSwan with full debugging
ipsec start --nofork --debug-all

# Verify network connectivity to server
ping 203.0.113.1

\end{lstlisting}

\subsection{Traffic Analysis}

Understanding how IPSec encapsulates and processes network traffic is crucial for verification and troubleshooting.

\subsubsection{Wireshark Packet Analysis}

To observe IPSec traffic in detail, it is highly recommended to open a Wireshark instance on one of the network cables connecting the client and server infrastructure. This allows real-time analysis of the VPN establishment and data transmission phases.

\noindent
\textbf{IPSec Tunnel Establishment Phase:}

\noindent
During the initial IPSec connection setup, Wireshark will capture ISAKMP (Internet Security Association and Key Management Protocol) packets. You can start the Wireshark capture instance by right clicking on a wire in GNS3, and then selecting "Start Capture". In this phase you will observe:
\begin{itemize}
    \item \textbf{ISAKMP packets on UDP port 500:} Initial key exchange and authentication
    \item \textbf{NAT-T packets on UDP port 4500:} NAT traversal negotiation if NAT is detected
    \item \textbf{Phase 1 and Phase 2 negotiations:} Security Association establishment
\end{itemize}

\noindent
\textbf{Active Tunnel Communication Phase:}

\noindent
Once the IPSec tunnel is established and active communication begins, the packet capture will show ESP (Encapsulating Security Payload) packets:

\begin{lstlisting}[language=bash]
# Filter for ESP traffic during active communication

# Test communication to generate ESP traffic
ping 192.168.30.1  # From client to SoftEther SecureNAT
\end{lstlisting}

\noindent
During active communication, you will observe:
\begin{itemize}
    \item \textbf{ESP packets:} Encrypted data transmission with visible ESP headers
    \item \textbf{Encrypted payload:} Data content protected by IPSec encryption
    \item \textbf{Tunnel endpoints:} Source and destination IPs showing the VPN gateway addresses
\end{itemize}

