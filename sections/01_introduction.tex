\newpage

\section{Introduction}

This Project involves the implementation and analysis of \textbf{Virtual Private Networks} (VPNs) using \textbf{SoftEther VPN} technology across an Internet infrastructure. VPNs are fundamental network security technologies that provide secure connectivity across shared, potentially untrusted networks by creating encrypted tunnels that preserve the confidentiality, integrity, and authenticity of data transmission.

\noindent
\\
The objective of this lab is to demonstrate how modern VPN technologies can establish secure communication channels between geographically separated networks, mimicking real-world scenarios in which branch offices or remote users need secure access to corporate resources. Through practical implementation, we will examine two different VPN protocols: IPSec and TLS/SSL, comparing their different approaches and methods to achieve network security.

\subsection{Laboratory Objectives}

The main objectives of this lab activity are:

\begin{itemize}
    \item \textbf{VPN Basics:} Learn the essentials of Virtual Private Networks and their application in modern network security architectures.
    
    \item \textbf{SoftEther VPN Deployment:} Install and deploy SoftEther VPN server to utilize multi-protocol VPN services, as IPSec and TLS-based connections.
    
    \item \textbf{Network Topology Design:} Design a realistic network topology using GNS3 that simulates Internet interconnectivity between two remote sites over an ISP infrastructure.
    
    \item \textbf{Protocol Comparison:} Contrast and compare different VPN protocols (IPSec vs. TLS) in terms of security features and complexity.
    
    \item \textbf{Traffic Analysis:} Perform packet capture with Wireshark to observe how differently the VPN protocols encapsulate and protect network traffic differently.
    
    \item \textbf{Security Assessment:} Examine the security mechanisms provided by each VPN implementation, including encryption, authentication, and integrity protection.
\end{itemize}

\subsection{Laboratory Environment and Network Topology}

This lab simulates a scenario in which two geographically separated private networks need to commmunicate securely over the Internet. The topology consists of:

\begin{itemize}
    \item \textbf{Server Site:} A private network (10.0.1.0/24) hosting the SoftEther VPN server
    \item \textbf{Client Site:} A remote private network (10.0.2.0/24) with VPN client capability
    \item \textbf{Internet Infrastructure:} Simulated ISP network providing connectivity between the sites
    \item \textbf{VPN Gateways:} Edge routers providing NAT and routing capabilities, to which a public IP address is assigned
\end{itemize}

\noindent
The configuration demonstrates how VPN technology facilitates secure site-to-site connections, allowing hosts whithin different private networks to communicate as if connected to the same local network, while preserving security attributes across the untrusted Internet infrastructure.

\subsection{System Requirements}

In order to successfully complete this lab activity, the following system requirements must be met:

\begin{tcolorbox}[colback=blue!5!white,colframe=blue!75!black,title=System Requirements]
\textbf{Operating System:}
\begin{itemize}
    \item Linux distribution (Ubuntu 20.04 LTS or newer recommended)
    \item Administrative (sudo) privileges required
\end{itemize}

\textbf{Software Requirements:}
\begin{itemize}
    \item \textbf{GNS3:} Graphical Network Simulator-3, a network simulation platform for simulating realistic network topologies
    \item \textbf{Docker:} Container system for VPN endpoints
    \item \textbf{Wireshark:} Network protocol analyzer
\end{itemize}

\textbf{Hardware Requirements:}
\begin{itemize}
    \item At least 4GB RAM (8GB highly recommended for high performance)
    \item 20GB free disk space
    \item Network interface with Internet connectivity
\end{itemize}
\end{tcolorbox}

\noindent
\textbf{Software Installation:}

\noindent
The following commands can be used to install the necessary software on Ubuntu/Debian systems:

\begin{lstlisting}[language=bash]
# Install GNS3 and dependencies
sudo apt install gns3-gui gns3-server

# Install Docker
sudo apt install docker.io

# Install network analysis tools
sudo apt install wireshark

# Add user to required groups
sudo usermod -aG docker $USER
sudo usermod -aG wireshark $USER
\end{lstlisting}

Note: An install and login cycle may be required after installation in order for group membership changes to take effect. The VPN client programs (strongSwan for IPSec and OpenVPN for TLS) will be pre-installed inside the Docker containers during lab setup.

\subsection{Lab Structure}

The report is organized into these sections:

\begin{enumerate}
    \item \textbf{Theoretical Background:} Introduction to VPN technologies, SoftEther VPN architecture, and employed security protocols.
    
    \item \textbf{Network Topology:} Schematic network description employed during simulation and IP addressing strategy.
    
    \item \textbf{Initial Configuration:} Setup of the GNS3 environment, the routers, and the container infrastructure.
    
    \item \textbf{SoftEther VPN Server:} Installation of the VPN server with multi-protocol support.
    
    \item \textbf{IPSec Configuration:} StrongSwan-based IPSec-based VPN implementation.
    
    \item \textbf{TLS Configuration:} OpenVPN client-based TLS/SSL VPN implementation.
        
\end{enumerate}

Each section contains in-practice implementation steps, configuration examples, and result analysis, providing a thorough overview of VPN technologies and applications in real-world contexts.

